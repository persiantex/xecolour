\documentclass[12pt]{xepersian-user}
\usepackage{xecolour}
\usepackage{supertabular}
\setmainfont[Scale=1,Mapping=tex-text]{Linux Libertine}
\def\Date{09/01/05}
\def\Version{0.1}
\makeindex
\begin{document}
\begingroup
\evensidemargin \paperwidth\relax
\advance\evensidemargin -2in
\advance\evensidemargin -\textwidth
\divide\evensidemargin 2
\oddsidemargin\evensidemargin
\begin{titlepage}

\let\footnotesize\small
\let\footnoterule\relax
\setcounter{page}{0}

\null
\vfil
\vskip 25pt
\begin{center}
  {\LARGE\textbf{\textsf{xecolour} Package}}\\[5pt]
  {\large\textbf{Colour in \XeLaTeX}.}\\
\end{center}

\bigskip\bigskip
\hbox to \hsize{%
  \hss
%\includegraphics[width=9cm,height=9cm]{iran.jpg}
   \hss}
\bigskip
\bigskip

\begin{center}
{\LARGE\textbf{User's Guide}}\par
\vskip 3em
{\large \lineskip .75em Vafa Khalighi}\par
\vskip 1.5em
{\large \thefiledate\\[2pt]
 Version \Version%
 \footnote{Documentation edited and repacked at \thefiledate\ by 
           Vafa Khalighi \url{vafa.khalighi@students.mq.edu.au}.}}\par

\end{center}
\vfil
\small
Author's address:\\
  4/34 Sorrell Street\\
  Parramatta\\
  NSW 2150\\
 Australia\\[3mm]
  Internet: \Verb[codes={\catcode`\<=12}]+<vafa.khalighi@students.mq.edu.au>+
\end{titlepage}

\endgroup

\setcounter{footnote}{0}
\tableofcontents
\pagenumbering{arabic}
\part*{Introduction}
As the author of \XePersian\ package (a package for Persian typesetting in \LaTeX\ over \XeTeX), I always wanted to write a package that make writing in colours easy in bidirectional texts. However there are packages \textsf{color} and \textsf{xcolor}, but writing bidirectional texts in different colours using these packages is a nightmare and the colours of the text will get mixed. In fact, the \textsf{bidi} package has a patch where one could use the packages mentioned above to write text in colour but unfortunately that text has got to be not more than a line, otherwise the colour of the text gets mixed.

Thus, I wrote \textsf{xecolour} package which allows the user to write texts in 140 different colours without any problem. I have used \XeTeX's feature to define 140 colours and I believe these 140 colours are all one needs. The command names are just the names of colours (i.e. "\blue" for blue colour) and the general syntax is:
\begingroup
\catcode`\<=12
\Mac \colourname{<text>}
\endgroup
In the next section, you will see a sample of each colour, and at the end of this manual, in the index part, all commands are written in alphabetical order. So if you wish to use a specific colour, please look at the index and find the command you should use.
\part{Sample Colours}
\begin{tabular}{rl}
\aliceblue{xecolour}&\n\aliceblue\\
\aquamarine{xecolour}&\n\aquamarine\\
\bisque{xecolour}&\n\bisque\\
\blue{xecolour} &\n\blue\\
\burlywood{xecolour}&\n\burlywood\\
\chocolate{xecolour}&\n\chocolate\\
\cornsilk{xecolour}&\n\cornsilk\\
\darkblue{xecolour}&\n\darkblue\\
\darkgray{xecolour}&\n\darkgray\\
\darkmagenta{xecolour}&\n\darkmagenta\\
\darkorchid{xecolour}&\n\darkorchid\\
\darkseagreen{xecolour}&\n\darkseagreen\\
\darkturquoise{xecolour}&\n\darkturquoise\\
\deepskyblue{xecolour}&\n\deepskyblue\\
\firebrick{xecolour}&\n\firebrick\\
\fuchsia{xecolour}&\n\fuchsia\\
\gold{xecolour}&\n\gold\\
\green{xecolour}&\n\green\\
\hotpink{xecolour}&\n\hotpink\\
\ivory{xecolour}&\n\ivory\\
\lavenderblush{xecolour}&\n\lavenderblush\\
\lightblue{xecolour}&\n\lightblue\\
\lightgoldenrodyellow{xecolour}&\n\lightgoldenrodyellow\\
\lightpink{xecolour}&\n\lightpink\\
\lightskyblue{xecolour}&\n\lightskyblue\\
\lightyellow{xecolour}&\n\lightyellow\\
\linen{xecolour}&\n\linen\\
\mediumaquamarine{xecolour}&\n\mediumaquamarine\\
\mediumpurple{xecolour}&\n\mediumpurple\\
\end{tabular}
\newpage
\begin{tabular}{rl}
\mediumspringgreen{xecolour}&\n\mediumspringgreen\\
\midnightblue{xecolour}&\n\midnightblue\\
\moccasin{xecolour}&\n\moccasin\\
\oldlace{xecolour}&\n\oldlace\\
\orangered{xecolour}&\n\orangered\\
\palegreen{xecolour}&\n\palegreen\\
\papayawhip{xecolour}&\n\papayawhip\\
\pink{xecolour}&\n\pink\\
\purple{xecolour}&\n\purple\\
\royalblue{xecolour}&\n\royalblue\\
\sandybrown{xecolour}&\n\sandybrown\\
\sienna{xecolour}&\n\sienna\\
\slateblue{xecolour}&\n\slateblue\\
\steelblue{xecolour}&\n\steelblue\\
\thistle{xecolor}&\n\thistle\\
\violet{xecolour}&\n\violet\\
\whitesmoke{xecolour}&\n\whitesmoke\\
\antiquewhite{xecolour}&\n\antiquewhite\\
\azure{xecolour}&\n\azure\\
\black{xecolour}&\n\black\\
\blueviolet{xecolour}&\n\blueviolet\\
\cadetblue{xecolour}&\n\cadetblue\\
\coral{xecolour}&\n\coral\\
\crimson{xecolour}&\n\crimson\\
\darkcyan{xecolour}&\n\darkcyan\\
\darkgreen{xecolour}&\n\darkgreen\\
\darkolivegreen{xecolour}&\n\darkolivegreen\\
\darkred{xecolour}&\n\darkred\\
\darkslateblue{xecolour}&\n\darkslateblue\\
\darkviolet{xecolour}&\n\darkviolet\\
\dimgray{xecolour}&\n\dimgray\\
\floralwhite{xecolour}&\n\floralwhite\\
\gainsboro{xecolour}&\n\gainsboro\\
\goldenrod{xecolour}&\n\goldenrod\\
\greenyellow{xecolour}&\n\greenyellow\\
\indianred{xecolour}&\n\indianred\\
\khaki{xecolour}&\n\khaki\\
\lawngreen{xecolour}&\n\lawngreen\\
\end{tabular}
\newpage
\begin{tabular}{rl}
\lightcoral{xecolour}&\n\lightcoral\\
\lightgreen{xecolour}&\n\lightgreen\\
\lightsalmon{xecolour}&\n\lightsalmon\\
\lightslategray{xecolour}&\n\lightslategray\\
\lime{xecolour}&\n\lime\\
\magenta{xecolour}&\n\magenta\\
\mediumblue{xecolour}&\n\mediumblue\\
\mediumseagreen{xecolour}&\n\mediumseagreen\\
\mediumturquoise{xecolour}&\n\mediumturquoise\\
\mintcream{xecolour}&\n\mintcream\\
\navajowhite{xecolour}&\n\navajowhite\\
\olivedrab{xecolour}&\n\olivedrab\\
\orchid{xecolour}&\n\orchid\\
\paleturquoise{xecolour}&\n\paleturquoise\\
\peachpuff{xecolour}&\n\peachpuff\\
\plum{xecolour}&\n\plum\\
\red{xecolour}&\n\red\\
\saddlebrown{xecolour}&\n\saddlebrown\\
\seagreen{xecolour}&\n\seagreen\\
\silver{xecolour}&\n\silver\\
\snow{xecolour}&\n\snow\\
\tancolour{xecolour}&\n\tancolour\\
\tomato{xecolour}&\n\tomato\\
\wheat{xecolour}&\n\wheat\\
\yellow{xecolour}&\n\yellow\\
\aqua{xecolour}&\n\aqua\\
\beige{xecolour}&\n\beige\\
\blanchedalmond{xecolour}&\n\blanchedalmond\\
\brown{xecolour}&\n\brown\\
\chartreuse{xecolour}&\n\chartreuse\\
\cornflowerblue{xecolour}&\n\cornflowerblue\\
\cyan{xecolour}&\n\cyan\\
\darkgoldenrod{xecolour}&\n\darkgoldenrod\\
\darkkhaki{xecolour}&\n\darkkhaki\\
\darkorange{xecolour}&\n\darkorange\\
\darksalmon{xecolour}&\n\darksalmon\\
\darkslategray{xecolour}&\n\darkslategray\\
\end{tabular}
\newpage
\begin{tabular}{rl}
\deeppink{xecolour}&\n\deeppink\\
\dodgerblue{xecolour}&\n\dodgerblue\\
\forestgreen{xecolour}&\n\forestgreen\\
\ghostwhite{xecolour}&\n\ghostwhite\\
\gray{xecolour}&\n\gray\\
\honeydew{xecolour}&\n\honeydew\\
\indigo{xecolour}&\n\indigo\\
\lavender{xecolour}&\n\lavender\\
\lemonchiffon{xecolour}&\n\lemonchiffon\\
\lightcyan{xecolour}&\n\lightcyan\\
\lightgrey{xecolour}&\n\lightgrey\\
\lightseagreen{xecolour}&\n\lightseagreen\\
\lightsteelblue{xecolour}&\n\lightsteelblue\\
\limegreen{xecolour}&\n\limegreen\\
\maroon{xecolour}&\n\maroon\\
\mediumorchid{xecolour}&\n\mediumorchid\\
\mediumslateblue{xecolour}&\n\mediumslateblue\\
\mediumvioletred{xecolour}&\n\mediumvioletred\\
\mistyrose{xecolour}&\n\mistyrose\\
\navy{xecolour}&\n\mistyrose\\
\orange{xecolour}&\n\orange\\
\palegoldenrod{xecolour}&\n\palegoldenrod\\
\palevioletred{xecolour}&\n\palevioletred\\
\peru{xecolour}&\n\peru\\
\powderblue{xecolour}&\n\powderblue\\
\rosybrown{xecolour}&\n\rosybrown\\
\salmon{xecolour}&\n\salmon\\
\seashell{xecolour}&\n\seashell\\
\skyblue{xecolour}&\n\skyblue\\
\springgreen{xecolour}&\n\springgreen\\
\teal{xecolour}&\n\teal\\
\turquoise{xecolour}&\n\turquoise\\
\white{xecolour}&\n\white\\
\yellowgreen{xecolour}&\n\yellowgreen\\
\lightyellow{xecolour}&\n\lightyellow\\
\crimson{xecolour}&\n\crimson\\
\bluishgreenishgrey{xecolour}&\n\bluishgreenishgrey\\
\end{tabular}



\PrintUserIndex
\end{document}
